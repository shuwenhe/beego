\documentclass[12pt,oneside,a4paper]{ctexbook} %oneside一面,twiside两面
\usepackage[centertags]{amsmath}
\usepackage{amsfonts}
\usepackage{amsthm}
\usepackage{newlfont}
\usepackage{makeidx}
\usepackage{wasysym}
\usepackage{geometry}%页边距和页眉页脚
\usepackage{graphics} %插图
\usepackage{extarrows}
\usepackage{amssymb}%因为所以
\usepackage{titlesec}%TEX标题与正文间距,标题与上下文距离调整
\usepackage{fancyhdr}%页眉
\usepackage{ctex}
\usepackage{listings} % 插入代码

\renewcommand\thesection{\arabic {section}}%去0
\titleformat{\chapter}[display]{\normalfont\huge\bfseries\center}{\chaptertitlename}{1pt}{\Huge}
\titleformat{\section}{\normalfont\Large\bfseries}{\thesection}{1em}{}
\titleformat{\subsection}{\normalfont\large\bfseries}{\thesubsection}{1em}{}
\titleformat{\subsubsection}{\normalfont\normalsize\bfseries}{\thesubsubsection}{1em}{}
\titleformat{\paragraph}[runin]{\normalfont\normalsize\bfseries}{\theparagraph}{1em}{}
\titleformat{\subparagraph}[runin]{\normalfont\normalsize\bfseries}{\thesubparagraph}{1em}{}
\titlespacing*{\chapter} {0pt}{10pt}{10pt}
\titlespacing*{\section} {0pt}{0.5ex plus 1ex minus .2ex}{0.3ex plus .2ex}
\titlespacing*{\subsection} {0pt}{0.25ex plus 1ex minus .1ex}{0.5ex plus .1ex}
\titlespacing*{\subsubsection}{0pt}{3.25ex plus 1ex minus .2ex}{1.5ex plus .2ex}
\titlespacing*{\paragraph} {0pt}{3.25ex plus 1ex minus .2ex}{1em}
\titlespacing*{\subparagraph} {\parindent}{3.25ex plus 1ex minus .2ex}{1em}
\numberwithin{chapter}{part}
\setlength\parskip{\baselineskip}% 增加空行%
\geometry{left=2.0cm,right=2.1cm,top=1.5cm,bottom=2.5cm}% 页边距和页眉页脚
\renewcommand\thesection{\arabic {section}} %目录需要从1-n

\lstset{language=C++}%这条命令可以让LaTeX排版时将C++ 键字突出显示
\lstset{breaklines}%这条命令可以让LaTeX自动将长的代码行换行排版
\lstset{extendedchars=false}%这一条命令可以解决代码跨页时,章节标题,页眉等汉字不显示的问题
\lstset{
    numbers=left,
    numberstyle= \tiny,
    keywordstyle= \color{ blue!70},
    commentstyle= \color{red!50!green!50!blue!50},
    rulesepcolor= \color{ red!20!green!20!blue!20} ,
    escapeinside=``, % 英文分号中可写入中文
    xleftmargin=2em,xrightmargin=2em, aboveskip=1em,
    framexleftmargin=2em,
}

\begin{document}

\pagestyle{fancy}
\lhead{\includegraphics{shuwenedu.png}}
\chead{}
\rhead{\includegraphics{xstiku.png}}
\lfoot{}
\rfoot{}

日期:\underline{\hbox to 32mm{\renewcommand{\today}{\number\year 年 \number\month 月 \number\day 日}\today}}
\qquad 姓名:\underline{\hbox to 25mm{}} \qquad 分数:\underline{\hbox to 20mm{}}

\begin{enumerate}
\item 
编号依次为1、2、3、$\cdots$、2013的2013盏电灯,各由一个拉线开关控制着.先将编号为偶数的电灯拉一下,再将编号为3的倍数的电灯拉一下,最后将编号为5的倍数的电灯拉一下.假设最初所有灯都亮着,那么经过上述操作,还有几盏灯是亮的呢?\\
解:\\
灯亮的=没拉的+拉两次的\\
拉过的\\$[\frac{2013}{2}]+[\frac{2013}{3}]+[\frac{2013}{5}]-[\frac{2013}{6}]-[\frac{2013}{10}]-[\frac{2013}{15}]+[\frac{2013}{30}]\\
=1006+671+402-335-201-134+67=1476$\\
没拉的$2013-1476=537$\\
拉2次$335+201+134-67\times3=469$\\
亮$=537+469=1006$

\item 
如图等腰三角形的面积是240平方厘米,$AB=AC=30$厘米,$D,E$为$AB$上两点,且$AD=15$厘米,$DE=10$厘米;$F,G$是$AC$上两点,且$AF=5$厘米,
$FG=10$厘米,那么阴影部分面积是\underline{\hbox to 10mm{}}.\\
\includegraphics{204.png}\\
解:\\
$S_{\triangle ADF}=\frac{1}{2}\times\frac{1}{6}\times240=20$\\
$S_{\triangle DEG}=\frac{10}{25}\times\frac{25}{30}\times\frac{15}{30}\times240=40$\\
$S_{\triangle BCE}=\frac{5}{30}\times240=40$\\
$20+40+40=100$

\item 
如图,大小两圆的相交部分(即阴影区域)的面积是大圆面积的$\frac{4}{15}$,是小圆面积的$\frac{3}{5}$.如果量得小圆的半径是5厘米,那么大圆半径是\underline{\hbox to 10mm{}} 厘米。\\
\includegraphics{10}\\
解:\\
小圆的面积为$\pi\times5^2=25\pi$,则大小u相交部分面积为$25\pi\times\frac{3}{5}= 15\pi$,那么大圆的面积为
$15\pi\div\frac{4}{15}= \frac{225}{4}\pi$,而$\frac{225}{4}=\frac{15}{2}\times\frac{15}{2}$,所以大圆半径为7.5厘米.

\item 
将所有的两位数从小到大连续写成一个多位数:$1011121314\cdots979899$,求这个多位数除以7的余数.\\
解:\\
$1011121314\cdots979899$\\
$=99+98\times10^2+97\times10^4+96\times10^6+\cdots+10\times10^{178}$\\
$\equiv(99+96+93+\cdots+12)+(98+95+\cdots+11)\times2+(97+94+\cdots+10)\times4$\\
$\equiv111\times15+109\times30+107\times60$\\
$\equiv6\times1+4\times2+2\times4$\\
$\equiv6+1+1$\\
$\equiv1(mod 7)$

\item 
A在B的上游,其间有一点C,AC是147千米,水流速度是2.5千米每时,BC是92千米,由于水面较宽阔,水流速度是1千米每时,甲在静水中的速度是18.5千米每时,乙在静水中的速度是22千米每时\\
(2)甲乙同时从A出发,乙在B处休息了20分钟,相遇地点距离点C处多少千米?\\
解:\\
(2)甲船由行驶到C$147\div(18.5+2.5)=7$小时\\
又经过$10-7+\frac{1}{3}=\frac{10}{3}$ 后\\
甲船又行驶了$(18.5+1)\times\frac{10}{3}=65$ 千米.\\
此时乙船的速度为$22-1=21$千米/小时.\\
甲、乙两船相距$92-65=27$千米\\
再经过$27\div(19.5+21)=\frac{2}{3}$小时相遇\\
此时距离C$(18.5+1)\times4=78$千米.

\item 
在隧道的入口处客车恰好追上了货车的车尾,在这条隧道的出口处客车又恰好超过了货车的车头,隧道长1680米,客车长120米,行驶速度为86.4千米/小时,货车速度75.6千米/小时。请问\\
(1)这列货车长多少米?(本题8分)\\
(2)这列货车通过隧道需要多长时间?(本题8分)\\
解:\\
(1)86.4千米/小时=24米/秒 \\
  75.6千米/小时=21米/秒\\
  整个追及时间为(1680+120)÷24=75秒\\
 货车长为1680-21×75=105米\\
(2)货车通过隧道时间(1680+105)÷21=85秒

\item 
水果店有橘子和柚子两种水果,已知橘子的数量是柚子数量的35倍。而每个柚子的重量是每个橘子的20倍。柚子5元一斤,而橘子3元一斤。每天卖出柚子8斤,橘子12斤,若干天后,卖橘子的收入比卖柚子的少了20元,此时橘子的数量是柚子的40倍,若已知5个橘子重1斤,那么剩下的橘子和柚子还能卖多少钱?\\
解:\\
设柚子$x$个,橘子$35x$个.\\
卖了$20\div(5\times8-3\times12)=5$天.\\
$35x-12\times5\times5=40(x-8\times5\div4)\\
x=20$\\
剩下10个柚子,400个橘子\\
$5\times4\times10+3\times80=440$元

\item 
某餐厅有$A,B,C$三种套餐,价值之比是4:6:7,某天统计销售情况时,发现选择$A,B,C$
三种套餐的人数之比是$12:3:2$,而且$B$套餐的总收入比$C$套餐多460元,
那么这一天$A$套餐的总收入是多少元?\\
解:\\
$A$套餐的总收入是$460\div(6\times3-7\times2)\times4\times12=5520$元.

\item 
鸡兔同笼,腿的总数比头数的总数的4倍少4只,则鸡有\underline{\hbox to 10mm{}} 只\\
解:\\
设有$x$个头,鸡有$y$个.\\
$2y+4(x-y)=4x-4$\\
$y=2$

\item 
能既表示9个连续正整数的和又能表示成连续10个正整数的和,又能表示为11个正整数之和的最小正整数为\underline{\hbox to 10mm{}}.\\
解:\\
根据题意,设9个连续自然数为:\\
a,a+1,a+2…,a+8,相加整理后得9(a+4)\\
也是连续10个自然数的和:b,b+1…,b+9,和是5(2b+9)\\
也是11个自然数的和:c,c+1…,c+10,和是11(c+5)\\
所以这个自然数是5,9,11的公倍数5,9,11最小公倍数是495.\\
所以符合条件的最小正整是495.

\item 
A、B、C三件衬衫的价格打折前合计1040元,打折后合计948 元,已知A衬衫的打折幅度是9.5 折,B衬衫的打折幅度是9 折,C 衬衫的打折幅度8.75 折,打折前A、B两件衬衫的价格比是5:4。问打折前A衬衫的价格是\underline{\hbox to 10mm{}} 元。\\
答案:400元、320元、320元\\
解析:从题目条件可以得到,\\
1件A+1件B+1件C=1040元……⑴,\\
0.95件A+0.9件B+0.875件C=948元……⑵,\\
1件A:1件B=5:4……⑶\\
⑵-⑴×0.875,得到:\\
0.075件A+0.025件B=948-0.875×1040=38元,\\
即3件A+1件B=1520元……⑷,\\
代入⑶,可以得到,\\
1件A=400元,1件B=320元,\\
1件C=320元\\
所以打折前A衬衫的价格分别是400元、320元和320元。

\item 
生理统计学家证实了贪官容易患心脑血管方面的疾病。统计某国的580 名贪官和600 名清廉的官员,发现后者健康(指没患心脑血管方面的疾病的)人数比前者多出272 人,这些官员中患心脑血管方面的人一共444 人。那么两类官员中健康人数的百分比各是\underline{\hbox to 10mm{}},\underline{\hbox to 10mm{}}.\\
解:\\
设清官中有$x$名健康者,则贪官中有$(x-272)$名健康者.\\
按题意列方程有$x+(x-272)+444=580+600$\\
得$x=504$\\
从而清官中的健康率为$504\div600=84\%$\\
贪官中的健康率为$(504-272)\div580 =40\%$.

\item 
有9颗相同的糖,每天至少吃1颗,要4天吃完,有多少种吃法?\\
解:56\\
3个板插入8个空(9颗糖中间8个空)\\
$C_{8}^{3}$\\
隔板法:\\
(1)元素相同\\
(2)每组至少有1个元素\\
(3)组合问题

\item 
甲、乙两地间公路全长500千米,其中平路占$\frac{1}{5}$. 由甲地到乙地去,上山路程是下山路程的$\frac{2}{3}$. 一辆汽车从甲地到乙地共行了10小时,已知这辆汽车行上山路的速度比平路慢$\frac{1}{5}$, 行下山的速度比平路快$\frac{1}{5}$, 照这样计算:\\
(2)汽车在平路上的速度为每小时多少千米?\\
(3)汽车由乙地返回甲地,需要行驶多少小时?\\
解:\\
(2)设平路速度为$5x$则有\\
$\frac{100}{5x}+\frac{160}{4x}+\frac{240}{6x}=10$\\
解得$x=10,5x=50$\\
平路速度是每小时50千米.\\
(3)汽车由乙地返回甲地上山路程为240千米,下山路程为160 千米\\
$\frac{240}{40}+\frac{160}{60}+\frac{100}{50}=10\frac{2}{3}$ 小时\\
乙地返回甲地需要行驶$10\frac{2}{3}$小时.

\item 
甲工程队每工作6天后休息1天,乙工程队每工作5天后休息2 天,一件工程甲队单独做需要97 天,乙队需要75 天。如果两队合作,同时开工,需要\underline{\hbox to 10mm{}} 天完工?(结果保留整数,不足一天按一天计算).\\
解:\\
$97\div(6+1)=13\cdots6$甲工作时间$14\times6=84$ 天,甲工效$\frac{1}{84}$\\
$75\div(5+2)=10\cdots5$乙工作时间$11\times5=55$ 天,乙工效$\frac{1}{55}$\\
一个周期7天,甲乙共做$\frac{6}{84}+\frac{5}{55}=\frac{1}{14}+\frac{1}{11}=\frac{25}{154}$\\
$1\div\frac{25}{154}=6\frac{4}{25}$,6 个周期$6\times7=42$天完成$\frac{25}{154}\times6=\frac{150}{154}$\\
$(1-\frac{150}{154})\div(\frac{1}{84}+\frac{1}{55})=\frac{2}{77}\div\frac{139}{4620}=\frac{9240}{10703}\approx1$\\
答:如果两队合作,同时开工,需要43天完工

\item 
甲乙丙三人练习跑步,在AB两地间往返跑,C点为中点,DB 占AB 的$\frac{1}{10}$,甲先出发,甲到达C 后乙出发,乙到达C后丙出发,丙到D 时遇到从B点回来的乙,丙继续到达B 时追上甲.\\
\includegraphics{45.png}\\
(1)求乙丙的速度比\\
(2)求甲丙的速度比\\
(3)设AB共有720米,当甲和丙第三次相遇(追上或迎头相遇)时,乙走了多少米?\\
解:\\
(1)由题意可设$DB$长度为1份则$AB$ 长度为10份,乙从$C$ 处到$B$处返回到$D$ 处共走$5+1=6$ 份,此时丙从$A$ 到达$D$ 点共$5+4=9$ 份,根据时间一定路程与速度成正比.所以乙丙的速度比为$6:9=2:3$\\
(2)$\because$乙丙的速度比为$2:3$$\therefore$乙丙走完$AB$ 全程所需时间比为$3:2$.\\
设乙走完全程用3份时间,丙走完全程用2份时间,当甲到达中点$C$ 处时,乙开始出发1.5小时到达中点,此时甲继续从中点向前行驶1.5小时.此时丙与甲同时行驶2 小时到达$B$ 点,所以甲从中点$C$到达B点用时$1.5+2=3.5$小时,所以甲从$A$到$B$ 用时$(1.5+2)\times2=7$小时.\\
$\because$路程一定,速度与时间成反比\\
$\therefore$甲丙速度比为$2:7$\\
(3)$\because$甲走完全程时间为7,丙走完全程为2.\\
$\therefore$甲走完720米全程前,丙可走完超过3个全程\\
$\therefore$设甲之后走了$x$米.\\
$\therefore \frac{x}{2}=\frac{2\times720+x}{7}$\\
$x=576$\\
$\because$
V甲:V乙=3:7\\
$\therefore$乙之后走了$576\times\frac{7}{3}=1344$ 米\\
乙之前:先走了$\frac{720}{2}=360$米.\\
丙之后走全程720\\
同时乙则走$720\div\frac{2}{3}=480$米\\
$\therefore$乙之前走了$360+480=840$米\\
$\therefore$总共走了$1344+840=2184$米.

\item 
客车和货车从甲乙两地出发,客车每小时行40 千米,货车每小时行48 千米,他们相遇后继续前进。客车到乙地后立即返回,返回时速度每小时增加14千米,货车返回时速度每小时减少4 千米,他们第二次相遇地点和第一次相距21 千米.\\
(1)货车到甲后,客车与乙的距离占甲乙总距离的几分之几?\\
(2)甲乙相距多少?\\
解:\\
(1)客车与货车速度比为40:48=5:6设两车相遇时货车走了6 份客车走了5份则货车走余下的5 份时客车走了$\frac{5}{6}\times5=\frac{25}{6}$ 即客车距离乙地还有$6-\frac{25}{6}=\frac{11}{6}$. 占甲乙总距离的$\frac{11}{6}\div(5+6)=\frac{1}{6}$.\\
(2)设甲乙相距$x$千米\\
客车返回时速度为$40+14=54$千米/小时\\
货车返回时速度为$48-4=44$千米/小时\\
当货车到达甲地,客车距离乙地还有$\frac{1}{6}x$ 千米,则客车$\frac{x}{6}\div40=\frac{x}{240}$ 小时后到达乙地.\\
此时货车返回了$\frac{x}{240}\times44=\frac{11x}{60}$ 千米.\\
即货车与客车此时相距$x-\frac{11x}{60}=\frac{49x}{60}$ 千米.\\
即货车与客车$\frac{49x}{60}\div(54+44)=\frac{x}{120}$ 小时后相遇.\\
此时客车走了$\frac{x}{120}\times54=\frac{9x}{20}$ 千米\\
即$\frac{6x}{11}-\frac{9x}{20}=21$\\
解得$x=220$\\
答:货车到甲后,客车与乙的距离占甲乙总距离的$\frac{1}{6}$, 甲乙相距220 千米.

\item 
正方形ABCD 边长为10m,蓝精灵和红精灵在这里玩追逐游戏。蓝精灵在A 点出发,红精灵在B 点出发,这两点相距10m。 蓝精灵在前,红精灵在后,他们俩朝同一方向不停的走,围着正方形绕圈,蓝精灵的速度为2 米每分钟,每跑5 分钟就休息1分钟,红精灵的速度为2.5 米每分钟,每跑4 分钟就休息1分钟。请问:\\
(1)11分钟后蓝精灵和红精灵的距离是\underline{\hbox to 10mm{}}.\\
(2)蓝精灵和红精灵\underline{\hbox to 10mm{}} 分钟后会相遇.\\
解:\\
(1)$\because$11分钟蓝精灵走了$2\times(5+5)=20$ 米\\
红精灵走了$2.5\times(4+4+1)=22.5$米\\
$\therefore$11分钟后蓝精灵和红精灵的距离是$10+20-22.5=7.5$米.\\
(2)1个周期6分钟内\\
蓝精灵走了$2\times5=10$米\\
红精灵走了$2.5\times(4+1)=12.5$ 米\\
一个周期红精灵追上蓝精灵2.5米\\
追上10米要$10\div2.5=4$个周期\\
即$6\times4=24$分钟后蓝精灵追上红精灵两人相遇.

\item 
甲、乙两人骑自行车于同时同地出发,沿着圆形跑道按逆时针方向行驶,甲每分钟行驶跑道的$\frac{9}{8}$ 圈,乙每分钟行驶跑道的$\frac{24}{25}$ 圈,从出发时刻起,到他们同时回到出发地点,至少需要的时间是\underline{\hbox to 10mm{}}.\\
 (A)$66\frac{2}{3}$\qquad(B)$33\frac{2}{3}$\qquad(C)$66\frac{1}{3}$
 \qquad(D)$33\frac{1}{3}$\\
解:\\
甲跑一圈所用时间为$\frac{8}{9}$,乙跑一圈所用时间为$\frac{25}{24}$\\
$\because$根据“子同母反”原理\\
$\therefore$得$[\frac{8}{9},\frac{25}{24}]=\frac{[8,25]}{(9,24)}=\frac{200}{3}=66\frac{2}{3}$

\item 
汽车从甲地到乙地,如果速度比预定的每小时慢5 千米,则到达所花的时间将比预定的多$\frac{1}{8}$,如果速度比预定的增加1/3,则到达时间比预定的早1小时,甲、乙两地相距\underline{\hbox to 10mm{}} 千米.\\
解:设甲乙之间相距$x$千米,原速度为$y$千米/小时,则
\begin{equation}
  \left\{
   \begin{array}{c}
\frac{x}{y-5}=\frac{9x}{8y}\\
\frac{x}{y+\frac{y}{3}}=\frac{x}{y}-1
   \end{array}
   \right.
\end{equation}
解得
\begin{equation}
\left\{
\begin{array}{c}
y=45\\
x=180
\end{array}
\right.
\end{equation}
答:甲乙之间距离180千米.

\item 
某市出租车的车费计算方式如下:路程在3公里(含3公里)为12 元,超过3公里但不超过15 公里,每1公里2.5元,
超过15公里后,每1公里3.5元,例如:小明乘出租车4.2公里,应付车费$12+(4.2-3)\times2.5=15$\\
(1)若小张在该市乘坐出租车行驶10公里,应付车费多少元?\\
(2)若小李在该市乘出租车到达终点后,需付车费49元,则小李乘车行驶了多少公里?\\
(3)若小王在该市乘坐出租车到达终点付完车费后,发现车费恰好每公里平均3元,则小王乘坐
出租车行驶了多少公里?(即求出按多少公里进行收费的即可)\\
解:\\
(1)应付$12+(10-3)\times2.5=29.5$元\\
(2)前15公里付费$12+(15-3)\times2.5=42$元,还有$49-42=7$ 元,又走了$7\div3.5=2$ 公里,所以
49元可以行驶17公里.\\
(3)
(类一)若设所行路程$x$在超过3公里但不超过15公里时,可以得出\\
$3x=12+2.5(x-3)$\\
$x=9$\\
(类二)若设所行路程$x$超过15公里,可得\\
$3x=12+(15-3)\times2.5+(x-15)\times3.5\\
x=21$

\item 
写有0,1,2,3,4,5的卡片各一张,能组成多少个百位不是2的六位奇数?\\
解:\\
(法一)分为两类,若百位是0,则依次确定个位、百位及其余各位的数字选择,满足条件的六位数共有
$3\times1\times4\times3\times2\times1=72$个;若百位不是0,按照同样的顺序确定其余各位,
满足条件的六位数共有$3\times3\times3\times3\times2\times1=162$个.故两类合计
$72+162=234$个.\\
(法二)计算奇数的个数,先选择个位,奇数共有$3\times4\times4\times3\times2\times1=288$个,
其中百位是2的奇数有$3\times1\times3\times3\times2\times1=54$个,故符合题目要求的数共有
$288-54=234$个.

\item 
某校初中共有三个年级,已知:\\
(1)该校女生人数是男生的$\frac{5}{4}$倍;\\
(2)初一年级男生人数比女生少$\frac{1}{4}$;\\
(3)初二的总人数占全校的$\frac{1}{3}$,且其中$\frac{2}{5}$是女生;\\
(4)初三男生占全校人数的$\frac{1}{9}$;\\
那么初三女生人数占该校总人数的几分之几?\\
解:\\
(法一)设该校学生共9份可得到初三女生为$\frac{11}{5}$, 即\\
$\frac{\frac{11}{5}}{9}=\frac{11}{45}$\\
答:初三女生人数占该校总人数的$\frac{11}{45}$\\
(法二)解:设总人数为“1”,由条件1,男生占总人数的\\
$1\div(1+\frac{5}{4})=\frac{4}{9}$\\
女生占总人数的$\frac{5}{9}$\\
由条件3,初二女生占总人数的$\frac{1}{3}\times\frac{2}{5}=\frac{2}{15}$,\\
初二男生占$\frac{1}{3}\times\frac{3}{5}=\frac{1}{5}$\\
所以初一男生占总人数的$\frac{4}{9}-\frac{1}{9}-\frac{1}{5}=\frac{2}{15}$,\\
初一女生占总人数的$\frac{3}{15}\div(1-\frac{1}{4})=\frac{8}{45}$.\\
那么,初三女生应占总人数的$\frac{5}{9}-\frac{2}{15}-\frac{8}{45}=\frac{11}{45}$.

\item 
用1,2,3,4这4个数字组成6位数,可以有重复数字,但数字1必须出现奇数次,
那么这样的6位数一共有多少个?\\
解:\\
1次$C_{6}^{1}\times3^5=1458$\\
3次$C_{6}^{3}\times3^3=540$\\
5次$C_{6}^{5}\times3=18$\\
$1458+540+18=2016$

\item 
有一个整数,用它去除70、110、160得到的三个余数之和是50。 这个整数是\underline{\hbox to 10mm{}}.\\
解:$70+110+160-50=290$\\
$290=29\times5\times2$\\
所以这个整数是29.

\item 
有不足100个的苹果,如果是10个一堆,那么剩余9个;9个一堆,那么剩余8个;6 个一堆,剩余5个;5个一堆,剩余4 个;3个一堆,剩余2 个.求开始有多少个苹果.\\
解:\\
设开始有$x$个苹果.\\
由题意可知\\
$10|x+1$\\
$9|x+1$\\
$6|x+1$\\
$5|x+1$\\
$3|x+1$\\
$\therefore [10,9,6]|x+1$\\
$\therefore 90|x+1$\\
$\therefore x+1=90$\\
$\therefore x=89$\\
开始有89个苹果.

\item 
将自然数1,2,3,4,$\cdots$依次写下去得到一个数:123456789101112$\cdots$如果写到某个自然数时,
所组成的数恰好第一次能够被72整除,
那么这个自然数是多少?\\
解:\\
设一直写到$n$,得到的多位数才首次被72整除,
则$9|1+2+3+\cdots+n$
即$9|\frac{n(n+1)}{2}$
所以$9|n$或$9|(n+1)$
又$n$显然是偶数,所以$n$可能是8,18,26,36,44,$\cdots$
再逐个验证所写的数末三位是否为8的倍数.
可知$n$最小为36.

\item 
一个自然数除以19余9,除以23余7,那么这个自然数最小是?\\
解:\\
由题意可设这个自然数为\\
$19x+9$或$23y+7$\\
则有$19x+9=23y+7$\\
$19x=23y-2$\\
$x=\frac{23y-2}{19}=\frac{19y+4y-2}{19}$\\
$y+\frac{4y-2}{19}$\\
当$4y-2=2\times19$\\
$y$取最小值16\\
这个自然数最小为$10\times23+7=237$

\item 
若$\underbrace{66\cdots6}_{100}$能整除$\underbrace{44\cdots4}_{n}$,求$n$的最小值.\\
解:\\
$\underbrace{66\cdots6}_{100}=3\times\underbrace{22\cdots2}_{100}$\\
3与$\underbrace{22\cdots2}_{100}$互质\\
$\therefore 3|\underbrace{44\cdots4}_{n}$\\
$\therefore 3|4n$\\
$\therefore 3|n$\\
又$\underbrace{22\cdots2}_{100}|\underbrace{44\cdots4}_{n}$\\
$\therefore 100|n$\\
$\therefore n$最小为300.

\item 
连续掷立方体骰子三次,三次掷得的最大点数恰好为5点的概率是多少?\\
解:\\
$A$三次最大点数不大于5,$P(A)=(\frac{5}{6})^3=\frac{125}{216}$\\
$B$三次最大点数不大于4,$P(B)=(\frac{4}{6})^3=\frac{64}{216}$\\
$\frac{125}{216}-\frac{64}{216}=\frac{61}{216}$

\item 
120乘以一个五位数以后,乘积是一个整数的立方,那么满足条件的五位数有多少个?\\
解:\\
$120=2^3\times3\times5$\\
所以另外一个乘数是$3^2\times5^2\times k^3$,$k$是整数,\\
所以$10000\leq3^2\times5^2\times k^3<100000$\\
$45 \leq k^3 \leq 444$\\
所以$k=4,5,6,7$\\
所以满足条件的五位数有4种可能.

\item 
设$N$是自然数,$\frac{N}{2}$是一个整数的平方数,$\frac{N}{5}$是一个整数的五次
方,那么$N$最小是多少?\\
解:\\
设$N=2^a5^b$\\
$\frac{N}{2}=2^{a-1}5^b$是平方数,所以$a-1$和$b$都是偶数,\\
$\frac{N}{5}=2^a5^{b-1}$是五次方数,所以$a$和$b-1$都是5的倍数,\\
所以$a$最小为5,$b$最小为6.\\
$N$最小为$2^5\times5^6=500000$

\item 
四个好朋友的年龄刚好是4个连续的自然数,4人年龄的乘积是11880,则这四个同学的年龄各是多少?\\
解:\\
$11880=2^3\times3^3\times5\times11=9\times10\times11\times12$

\item 
有6个数字$a,b,c,d,p,q$满足$\overline{abc}-\overline{bcd}=p\times q\times\overline{pq}$\\
$c$和$d$的奇偶性相同,且$p,q,\overline{pq}$都是质数,则$\overline{abcd}$ 最大是多少?\\
解:\\
因为$c$和$d$奇偶性相同,所以$\overline{abc}-\overline{bcd}$是偶数,\\
所以$p\times q\times \overline{pq}$是偶数.由此易知$p=2,q=3$\\
所以$\overline{abc}-\overline{bcd}=2\times3\times23=138$\\
$\overline{abcd}$最大为9846.

\item 
三个不同质数$a,b,c$满足$ab^bc+a=2000$求$a+b+c$.\\
解:\\
$a|2000,a=2$或5\\
(1)若$a=2,2b^bc+2=2000,b^bc=999=3^3\times37,b=3,c=37$\\
(2)若$a=5,5b^bc+5=2000,b^bc=399=3\times7\times19,b,c$不存在.\\
所以$a=2,b=3,c=37,a+b+c=2+3+37=42$.

\item 
设$p$和$q$均为自然数,且
$\frac{p}{q}=1-\frac{1}{2}+\frac{1}{3}-\frac{1}{4}+\frac{1}{5}-\cdots-\frac{1}{18}+\frac{1}{19}$\\
求证:$29|p$.\\
证:\\
$\frac{p}{q}=1-\frac{1}{2}+\frac{1}{3}-\frac{1}{4}+\frac{1}{5}-\cdots-\frac{1}{18}+\frac{1}{19}$\\
$=1+\frac{1}{2}+\frac{1}{3}+\frac{1}{4}+\frac{1}{5}+\cdots+\frac{1}{19}-\frac{2}{2}-\frac{2}{4}-\frac{2}{6}-\cdots-\frac{2}{18}$\\
$=\frac{1}{10}+\frac{1}{11}+\frac{1}{12}+\cdots+\frac{1}{19}$\\
$=(\frac{1}{10}+\frac{1}{19})+(\frac{1}{11}+\frac{1}{18})+(\frac{1}{12}+\frac{1}{17})+(\frac{1}{13}+\frac{1}{16})+(\frac{1}{14}+\frac{1}{15})$\\
$=29\times(\frac{1}{10\times19}+\frac{1}{11\times18}+\frac{1}{12\times17}+\frac{1}{13\times16}+\frac{1}{14\times15})$\\
$\therefore 29|p$

\item 
有三个自然数最大的比最小的大6,另一个是他们的平均数,且三个数的乘积是42560,求这三个自然数.\\
解:\\
注意$30^3=27000<42560<40^3=64000$所以这3个数均在30到40之间,
且通过分解质因数得到$42560=2^6\times5\times7\times19$,由此易知这3个数为32、35、38.

\item 
如果两数的和是64,两数的积可以整除4875,那么这两个数的差等于多少?\\
解:\\
$4875=3\times5\times5\times5\times13$,所以这两个数为39、25,差为14.

\end{enumerate}
\end{document}